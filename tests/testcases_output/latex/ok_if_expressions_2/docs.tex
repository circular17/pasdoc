\documentclass{report}
\usepackage{hyperref}
% WARNING: THIS SHOULD BE MODIFIED DEPENDING ON THE LETTER/A4 SIZE
\oddsidemargin 0cm
\evensidemargin 0cm
\marginparsep 0cm
\marginparwidth 0cm
\parindent 0cm
\setlength{\textwidth}{\paperwidth}
\addtolength{\textwidth}{-2in}


% Conditional define to determine if pdf output is used
\newif\ifpdf
\ifx\pdfoutput\undefined
\pdffalse
\else
\pdfoutput=1
\pdftrue
\fi

\ifpdf
  \usepackage[pdftex]{graphicx}
\else
  \usepackage[dvips]{graphicx}
\fi

% Write Document information for pdflatex/pdftex
\ifpdf
\pdfinfo{
 /Author     (Pasdoc)
 /Title      ()
}
\fi


% definitons for warning and note tag
\usepackage[most]{tcolorbox}
\newtcolorbox{tcbwarning}{
 breakable,
 enhanced jigsaw,
 top=0pt,
 bottom=0pt,
 titlerule=0pt,
 bottomtitle=0pt,
 rightrule=0pt,
 toprule=0pt,
 bottomrule=0pt,
 colback=white,
 arc=0pt,
 outer arc=0pt,
 title style={white},
 fonttitle=\color{black}\bfseries,
 left=8pt,
 colframe=red,
 title={Warning:},
}
\newtcolorbox{tcbnote}{
 breakable,
 enhanced jigsaw,
 top=0pt,
 bottom=0pt,
 titlerule=0pt,
 bottomtitle=0pt,
 rightrule=0pt,
 toprule=0pt,
 bottomrule=0pt,
 colback=white,
 arc=0pt,
 outer arc=0pt,
 title style={white},
 fonttitle=\color{black}\bfseries,
 left=8pt,
 colframe=yellow,
 title={Note:},
}

\begin{document}
\label{toc}\tableofcontents
\newpage
% special variable used for calculating some widths.
\newlength{\tmplength}
\chapter{Unit ok{\_}if{\_}expressions{\_}2}
\label{ok_if_expressions_2}
\index{ok{\_}if{\_}expressions{\_}2}
\section{Constants}
\ifpdf
\subsection*{\large{\textbf{{\_}TrueXorTrueIsFalse}}\normalsize\hspace{1ex}\hrulefill}
\else
\subsection*{{\_}TrueXorTrueIsFalse}
\fi
\label{ok_if_expressions_2-_TrueXorTrueIsFalse}
\index{{\_}TrueXorTrueIsFalse}
\begin{list}{}{
\settowidth{\tmplength}{\textbf{Description}}
\setlength{\itemindent}{0cm}
\setlength{\listparindent}{0cm}
\setlength{\leftmargin}{\evensidemargin}
\addtolength{\leftmargin}{\tmplength}
\settowidth{\labelsep}{X}
\addtolength{\leftmargin}{\labelsep}
\setlength{\labelwidth}{\tmplength}
}
\item[\textbf{Declaration}\hfill]
\ifpdf
\begin{flushleft}
\fi
\begin{ttfamily}
{\_}TrueXorTrueIsFalse =

true
;\end{ttfamily}

\ifpdf
\end{flushleft}
\fi

\end{list}
\ifpdf
\subsection*{\large{\textbf{{\_}3Plus1Is4}}\normalsize\hspace{1ex}\hrulefill}
\else
\subsection*{{\_}3Plus1Is4}
\fi
\label{ok_if_expressions_2-_3Plus1Is4}
\index{{\_}3Plus1Is4}
\begin{list}{}{
\settowidth{\tmplength}{\textbf{Description}}
\setlength{\itemindent}{0cm}
\setlength{\listparindent}{0cm}
\setlength{\leftmargin}{\evensidemargin}
\addtolength{\leftmargin}{\tmplength}
\settowidth{\labelsep}{X}
\addtolength{\leftmargin}{\labelsep}
\setlength{\labelwidth}{\tmplength}
}
\item[\textbf{Declaration}\hfill]
\ifpdf
\begin{flushleft}
\fi
\begin{ttfamily}
{\_}3Plus1Is4 =

true
;\end{ttfamily}

\ifpdf
\end{flushleft}
\fi

\end{list}
\ifpdf
\subsection*{\large{\textbf{{\_}APlusBIsAB}}\normalsize\hspace{1ex}\hrulefill}
\else
\subsection*{{\_}APlusBIsAB}
\fi
\label{ok_if_expressions_2-_APlusBIsAB}
\index{{\_}APlusBIsAB}
\begin{list}{}{
\settowidth{\tmplength}{\textbf{Description}}
\setlength{\itemindent}{0cm}
\setlength{\listparindent}{0cm}
\setlength{\leftmargin}{\evensidemargin}
\addtolength{\leftmargin}{\tmplength}
\settowidth{\labelsep}{X}
\addtolength{\leftmargin}{\labelsep}
\setlength{\labelwidth}{\tmplength}
}
\item[\textbf{Declaration}\hfill]
\ifpdf
\begin{flushleft}
\fi
\begin{ttfamily}
{\_}APlusBIsAB =

true
;\end{ttfamily}

\ifpdf
\end{flushleft}
\fi

\end{list}
\ifpdf
\subsection*{\large{\textbf{{\_}3Plus2Times2Is7}}\normalsize\hspace{1ex}\hrulefill}
\else
\subsection*{{\_}3Plus2Times2Is7}
\fi
\label{ok_if_expressions_2-_3Plus2Times2Is7}
\index{{\_}3Plus2Times2Is7}
\begin{list}{}{
\settowidth{\tmplength}{\textbf{Description}}
\setlength{\itemindent}{0cm}
\setlength{\listparindent}{0cm}
\setlength{\leftmargin}{\evensidemargin}
\addtolength{\leftmargin}{\tmplength}
\settowidth{\labelsep}{X}
\addtolength{\leftmargin}{\labelsep}
\setlength{\labelwidth}{\tmplength}
}
\item[\textbf{Declaration}\hfill]
\ifpdf
\begin{flushleft}
\fi
\begin{ttfamily}
{\_}3Plus2Times2Is7 =

true
;\end{ttfamily}

\ifpdf
\end{flushleft}
\fi

\end{list}
\ifpdf
\subsection*{\large{\textbf{{\_}3Minus1IsNot4}}\normalsize\hspace{1ex}\hrulefill}
\else
\subsection*{{\_}3Minus1IsNot4}
\fi
\label{ok_if_expressions_2-_3Minus1IsNot4}
\index{{\_}3Minus1IsNot4}
\begin{list}{}{
\settowidth{\tmplength}{\textbf{Description}}
\setlength{\itemindent}{0cm}
\setlength{\listparindent}{0cm}
\setlength{\leftmargin}{\evensidemargin}
\addtolength{\leftmargin}{\tmplength}
\settowidth{\labelsep}{X}
\addtolength{\leftmargin}{\labelsep}
\setlength{\labelwidth}{\tmplength}
}
\item[\textbf{Declaration}\hfill]
\ifpdf
\begin{flushleft}
\fi
\begin{ttfamily}
{\_}3Minus1IsNot4 =

true
;\end{ttfamily}

\ifpdf
\end{flushleft}
\fi

\end{list}
\ifpdf
\subsection*{\large{\textbf{{\_}3Minus2IsTrue}}\normalsize\hspace{1ex}\hrulefill}
\else
\subsection*{{\_}3Minus2IsTrue}
\fi
\label{ok_if_expressions_2-_3Minus2IsTrue}
\index{{\_}3Minus2IsTrue}
\begin{list}{}{
\settowidth{\tmplength}{\textbf{Description}}
\setlength{\itemindent}{0cm}
\setlength{\listparindent}{0cm}
\setlength{\leftmargin}{\evensidemargin}
\addtolength{\leftmargin}{\tmplength}
\settowidth{\labelsep}{X}
\addtolength{\leftmargin}{\labelsep}
\setlength{\labelwidth}{\tmplength}
}
\item[\textbf{Declaration}\hfill]
\ifpdf
\begin{flushleft}
\fi
\begin{ttfamily}
{\_}3Minus2IsTrue =

true
;\end{ttfamily}

\ifpdf
\end{flushleft}
\fi

\end{list}
\ifpdf
\subsection*{\large{\textbf{{\_}Not3Plus5IsTrue}}\normalsize\hspace{1ex}\hrulefill}
\else
\subsection*{{\_}Not3Plus5IsTrue}
\fi
\label{ok_if_expressions_2-_Not3Plus5IsTrue}
\index{{\_}Not3Plus5IsTrue}
\begin{list}{}{
\settowidth{\tmplength}{\textbf{Description}}
\setlength{\itemindent}{0cm}
\setlength{\listparindent}{0cm}
\setlength{\leftmargin}{\evensidemargin}
\addtolength{\leftmargin}{\tmplength}
\settowidth{\labelsep}{X}
\addtolength{\leftmargin}{\labelsep}
\setlength{\labelwidth}{\tmplength}
}
\item[\textbf{Declaration}\hfill]
\ifpdf
\begin{flushleft}
\fi
\begin{ttfamily}
{\_}Not3Plus5IsTrue =

true
;\end{ttfamily}

\ifpdf
\end{flushleft}
\fi

\par
\item[\textbf{Description}]
This should be true because NOT 3 = {-}4

\end{list}
\ifpdf
\subsection*{\large{\textbf{{\_}Not1Is0}}\normalsize\hspace{1ex}\hrulefill}
\else
\subsection*{{\_}Not1Is0}
\fi
\label{ok_if_expressions_2-_Not1Is0}
\index{{\_}Not1Is0}
\begin{list}{}{
\settowidth{\tmplength}{\textbf{Description}}
\setlength{\itemindent}{0cm}
\setlength{\listparindent}{0cm}
\setlength{\leftmargin}{\evensidemargin}
\addtolength{\leftmargin}{\tmplength}
\settowidth{\labelsep}{X}
\addtolength{\leftmargin}{\labelsep}
\setlength{\labelwidth}{\tmplength}
}
\item[\textbf{Declaration}\hfill]
\ifpdf
\begin{flushleft}
\fi
\begin{ttfamily}
{\_}Not1Is0 =

true
;\end{ttfamily}

\ifpdf
\end{flushleft}
\fi

\par
\item[\textbf{Description}]
This should be true because NOT as an exception swaps 0 and 1

\end{list}
\ifpdf
\subsection*{\large{\textbf{{\_}Not0Is1}}\normalsize\hspace{1ex}\hrulefill}
\else
\subsection*{{\_}Not0Is1}
\fi
\label{ok_if_expressions_2-_Not0Is1}
\index{{\_}Not0Is1}
\begin{list}{}{
\settowidth{\tmplength}{\textbf{Description}}
\setlength{\itemindent}{0cm}
\setlength{\listparindent}{0cm}
\setlength{\leftmargin}{\evensidemargin}
\addtolength{\leftmargin}{\tmplength}
\settowidth{\labelsep}{X}
\addtolength{\leftmargin}{\labelsep}
\setlength{\labelwidth}{\tmplength}
}
\item[\textbf{Declaration}\hfill]
\ifpdf
\begin{flushleft}
\fi
\begin{ttfamily}
{\_}Not0Is1 =

true
;\end{ttfamily}

\ifpdf
\end{flushleft}
\fi

\par
\item[\textbf{Description}]
This should be true because NOT as an exception swaps 0 and 1

\end{list}
\ifpdf
\subsection*{\large{\textbf{{\_}2Or4Is6}}\normalsize\hspace{1ex}\hrulefill}
\else
\subsection*{{\_}2Or4Is6}
\fi
\label{ok_if_expressions_2-_2Or4Is6}
\index{{\_}2Or4Is6}
\begin{list}{}{
\settowidth{\tmplength}{\textbf{Description}}
\setlength{\itemindent}{0cm}
\setlength{\listparindent}{0cm}
\setlength{\leftmargin}{\evensidemargin}
\addtolength{\leftmargin}{\tmplength}
\settowidth{\labelsep}{X}
\addtolength{\leftmargin}{\labelsep}
\setlength{\labelwidth}{\tmplength}
}
\item[\textbf{Declaration}\hfill]
\ifpdf
\begin{flushleft}
\fi
\begin{ttfamily}
{\_}2Or4Is6 =

true
;\end{ttfamily}

\ifpdf
\end{flushleft}
\fi

\end{list}
\ifpdf
\subsection*{\large{\textbf{{\_}6And4Is4}}\normalsize\hspace{1ex}\hrulefill}
\else
\subsection*{{\_}6And4Is4}
\fi
\label{ok_if_expressions_2-_6And4Is4}
\index{{\_}6And4Is4}
\begin{list}{}{
\settowidth{\tmplength}{\textbf{Description}}
\setlength{\itemindent}{0cm}
\setlength{\listparindent}{0cm}
\setlength{\leftmargin}{\evensidemargin}
\addtolength{\leftmargin}{\tmplength}
\settowidth{\labelsep}{X}
\addtolength{\leftmargin}{\labelsep}
\setlength{\labelwidth}{\tmplength}
}
\item[\textbf{Declaration}\hfill]
\ifpdf
\begin{flushleft}
\fi
\begin{ttfamily}
{\_}6And4Is4 =

true
;\end{ttfamily}

\ifpdf
\end{flushleft}
\fi

\end{list}
\ifpdf
\subsection*{\large{\textbf{{\_}2And6Is2}}\normalsize\hspace{1ex}\hrulefill}
\else
\subsection*{{\_}2And6Is2}
\fi
\label{ok_if_expressions_2-_2And6Is2}
\index{{\_}2And6Is2}
\begin{list}{}{
\settowidth{\tmplength}{\textbf{Description}}
\setlength{\itemindent}{0cm}
\setlength{\listparindent}{0cm}
\setlength{\leftmargin}{\evensidemargin}
\addtolength{\leftmargin}{\tmplength}
\settowidth{\labelsep}{X}
\addtolength{\leftmargin}{\labelsep}
\setlength{\labelwidth}{\tmplength}
}
\item[\textbf{Declaration}\hfill]
\ifpdf
\begin{flushleft}
\fi
\begin{ttfamily}
{\_}2And6Is2 =

true
;\end{ttfamily}

\ifpdf
\end{flushleft}
\fi

\end{list}
\ifpdf
\subsection*{\large{\textbf{FPCVersionAtLeast322}}\normalsize\hspace{1ex}\hrulefill}
\else
\subsection*{FPCVersionAtLeast322}
\fi
\label{ok_if_expressions_2-FPCVersionAtLeast322}
\index{FPCVersionAtLeast322}
\begin{list}{}{
\settowidth{\tmplength}{\textbf{Description}}
\setlength{\itemindent}{0cm}
\setlength{\listparindent}{0cm}
\setlength{\leftmargin}{\evensidemargin}
\addtolength{\leftmargin}{\tmplength}
\settowidth{\labelsep}{X}
\addtolength{\leftmargin}{\labelsep}
\setlength{\labelwidth}{\tmplength}
}
\item[\textbf{Declaration}\hfill]
\ifpdf
\begin{flushleft}
\fi
\begin{ttfamily}
FPCVersionAtLeast322 =

true
;\end{ttfamily}

\ifpdf
\end{flushleft}
\fi

\end{list}
\ifpdf
\subsection*{\large{\textbf{FPCVersionAtLeast422}}\normalsize\hspace{1ex}\hrulefill}
\else
\subsection*{FPCVersionAtLeast422}
\fi
\label{ok_if_expressions_2-FPCVersionAtLeast422}
\index{FPCVersionAtLeast422}
\begin{list}{}{
\settowidth{\tmplength}{\textbf{Description}}
\setlength{\itemindent}{0cm}
\setlength{\listparindent}{0cm}
\setlength{\leftmargin}{\evensidemargin}
\addtolength{\leftmargin}{\tmplength}
\settowidth{\labelsep}{X}
\addtolength{\leftmargin}{\labelsep}
\setlength{\labelwidth}{\tmplength}
}
\item[\textbf{Declaration}\hfill]
\ifpdf
\begin{flushleft}
\fi
\begin{ttfamily}
FPCVersionAtLeast422 =

false
;\end{ttfamily}

\ifpdf
\end{flushleft}
\fi

\end{list}
\ifpdf
\subsection*{\large{\textbf{MyConst}}\normalsize\hspace{1ex}\hrulefill}
\else
\subsection*{MyConst}
\fi
\label{ok_if_expressions_2-MyConst}
\index{MyConst}
\begin{list}{}{
\settowidth{\tmplength}{\textbf{Description}}
\setlength{\itemindent}{0cm}
\setlength{\listparindent}{0cm}
\setlength{\leftmargin}{\evensidemargin}
\addtolength{\leftmargin}{\tmplength}
\settowidth{\labelsep}{X}
\addtolength{\leftmargin}{\labelsep}
\setlength{\labelwidth}{\tmplength}
}
\item[\textbf{Declaration}\hfill]
\ifpdf
\begin{flushleft}
\fi
\begin{ttfamily}
MyConst = 30202;\end{ttfamily}

\ifpdf
\end{flushleft}
\fi

\par
\item[\textbf{Description}]
Note: This does not compile in FPC. But PasDoc, for now, unifies symbols used in {\$}if expressions with macros used during normal Pascal code parsing, so it actually replaces FPC{\_}FULLVERSION below.

\end{list}
\end{document}
